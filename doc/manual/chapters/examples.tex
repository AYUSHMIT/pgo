\subsection{Euclidean algorithm}
The Euclidean algorithm is a simple algorithm that computes the greatest common divisor of two integers, and is a good example PlusCal algorithm.

\lstinputlisting[language=pcal]{examples/Euclid.tla}
\captionof{figure}{PlusCal}

\lstinputlisting[language=golang]{examples/Euclid.go}
\captionof{figure}{Compiled Go}

The constant \texttt{N} needs to be annotated since its declaration does not appear in the comment containing the algorithm. Note the code to swap \texttt{u} and \texttt{v} in lines 26-29 of the Go program.

\subsection{N-Queens problem}
This PlusCal algorithm computes all possible ways to place $n$ queens on an $n \times n$ chessboard such that no two queens attack each other. It demonstrates the expressive power of PlusCal's set constructs, as the algorithm is very concise.

\lstinputlisting[language=pcal]{examples/Queens.tla}
\captionof{figure}{PlusCal}

\lstinputlisting[language=golang]{examples/Queens.go}
\captionof{figure}{Compiled Go}

(Some lines were broken in order to fit on the page.)

Note that the variables \texttt{cols} and \texttt{exts} do not require a type annotation, though they involve PlusCal tuples, since PGo can infer their type based on the type of \texttt{todo}.

\subsection{Dijkstra's mutex algorithm}
This is a multiprocess algorithm which only allows one process to be in the critical section at one time.

\lstinputlisting[language=pcal]{examples/DijkstraMutex.tla}
\captionof{figure}{PlusCal}

\lstinputlisting[language=golang]{examples/DijkstraMutex.go}
\captionof{figure}{Compiled Go}

The Go statements used in compilation are atomic on a usual configuration, so locking is turned off. (All \texttt{datatypes.Map} and \texttt{datatypes.Set} operations are atomic.)

For reference, here is the compiled code when locking is turned on:

\lstinputlisting[language=golang]{examples/DijkstraMutex-lock.go}
\captionof{figure}{Compiled Go with locks}

Note that the loop at line 44 does not require any locking, since it only involves local variables.

The constant \texttt{Proc} is defined to be \texttt{1 .. 10} in the annotation. If the process set needs to be changed, only the annotation needs to be edited.
