\subsection{Euclidean algorithm}
The Euclidean algorithm is a simple algorithm that computes the greatest common divisor of two integers, and is a good example PlusCal algorithm.

\lstinputlisting[language=pcal]{../../examples/Euclid.tla}
\captionof{figure}{PlusCal}

\lstinputlisting[language=golang]{../../examples/output/euclid.go}
\captionof{figure}{Compiled Go}

The constant \lstinline[language=pcal]|N| needs to be specified in the configuration file whose path is passed to PGo, since its definition does not appear in the comment containing the algorithm. Note the code to swap \lstinline[language=pcal]|u| and \lstinline[language=pcal]|v| on line 26 of the Go program.

\subsection{N-Queens problem}
This PlusCal algorithm computes all possible ways to place $n$ queens on an $n \times n$ chessboard such that no two queens attack each other. It demonstrates the expressive power of PlusCal's set constructs, as the algorithm is very concise.

\lstinputlisting[language=pcal]{../../examples/Queens.tla}
\captionof{figure}{PlusCal}

\lstinputlisting[language=golang]{../../examples/output/queens.go}
\captionof{figure}{Compiled Go}

Note that the non-trivial types for all variables are correctly inferred by PGo.

\subsection{Dijkstra's mutex algorithm}
This is a multiprocess algorithm which only allows one process to be in the critical section at one time.

\lstinputlisting[language=pcal]{../../examples/DijkstraMutex.tla}
\captionof{figure}{PlusCal}

\lstinputlisting[language=golang]{../../examples/output/dijkstra-mutex.go}
\captionof{figure}{Compiled Go}

The constant \lstinline[language=pcal]|Proc| is defined to be \lstinline[language=pcal]|1 .. 3| in the configuration. If the process set needs to be changed, only the configuration needs to be edited.
