\subsection{Installation}
Requirements: IntelliJ, Eclipse, or Ant 1.9

\begin{itemize}
\item Git clone the source at \url{https://github.com/UBC-NSS/pgo}

\item Option 1: Import as an IntelliJ project \\
Option 2: Import as an Eclipse project \\
Option 3: Execute \texttt{ant build} to compile the project and then execute \texttt{pgo.sh [options] pcalfile} to compile \texttt{pcalfile}.
\end{itemize}

Dependencies:
\begin{itemize}
\item The \href{https://mernst.github.io/plume-lib/}
	{Plume options library}.

\item \href{http://hamcrest.org/JavaHamcrest/}
	{Java Hamcrest}.

\item The \href{https://github.com/stleary/JSON-java}
	{JSON reference implementation}.
\end{itemize}


PGo was tested on JRE 8, JRE 9, and Go 1.10.

\subsection{Execution}
To run PGo, run the IntelliJ project, the Eclipse project or run \texttt{pgo.sh}. The command-line usage is \texttt{pgo [options] pcalfile}.

Optional command line arguments:

\begin{verbatim}
	--version=<boolean>          - Version [default false]
	-h --help=<boolean>          - Print usage information [default false]
	-q --logLvlQuiet=<boolean>   - Reduce printing during execution [default false]
	-v --logLvlVerbose=<boolean> - Print detailed information during execution  [default false]
	-m --mpcalCompile=<boolean>  - Compile a Modular PlusCal spec to vanilla PlusCal [default false]
	-c --configFilePath=<string> - path to the configuration file, if any
\end{verbatim}

\subsection{Configuration}
PGo requires a JSON configuration file with the following information.

\begin{lstlisting}
{
	"build": {
		"output_dir": "/path/to/output",
		"dest_file": "out.go"
	},
	"networking": {
		"enabled": true,
		"state": {
			"strategy": "state-server",
			"endpoints": ["10.0.0.1:1234", "10.0.0.2:1235"],
			"peers": ["10.0.0.3:4321", "10.0.0.4:4322"],
			"timeout": 3
		}
	},
	"constants": {
		"name": "value"
	}
}
\end{lstlisting}

\subsubsection{Build}
\noindent\lstinline|output_dir| must point to an existing directory.

\noindent\lstinline|dest_file| specifies the output Go file for PGo to write into. The full path for the file is constructed by appending the value of \lstinline|dest_file| to \lstinline|output_dir|. This file will be truncated by PGo.
\subsubsection{Networking}
\noindent\lstinline|enabled| specifies whether the compiled Go program is a distributed program backed by a network. \lstinline|enabled| must be \lstinline|false| when the input PlusCal file is a uniprocess algorithm, otherwise PGo will halt with an error.

\noindent\lstinline|state| specifies the strategy to use for distributed program compilation. It is ignored when \lstinline|enabled| is \lstinline|false|. Currently, \lstinline|etcd| and \lstinline|state-server| strategies are supported. The default strategy to use is \lstinline|state-server|.

\noindent\lstinline|peers| specifies a list of peers among which the distributed processes have to establish connections.

\noindent\lstinline|endpoints| specifies the etcd endpoints to which the distributed processes have to connect.

\noindent\lstinline|timeout| specifies the timeout interval in seconds. The default value for this option is 3 seconds.
\subsubsection{Constants}
The PlusCal algorithm can make use of TLA+ constants that are found outside the algorithm block (i.e. constants declared using the \lstinline[language=pcal]|CONSTANT| keyword). Concrete values for these constants need to be specified in the \lstinline|constants| dict. Each key is a JSON string containing the name of the constant being defined. Each value is a JSON string containing one valid TLA+ expression.

\noindent\begin{minipage}{0.45\textwidth}
\begin{lstlisting}
{
  ...
  "constants": {
    "myProcs": "{1, 3}",
    "N": "3"
  },
  ...
}
\end{lstlisting}
\captionof{figure}{Example constant specification}
\end{minipage}
\hfill\begin{minipage}{0.45\textwidth}
\begin{lstlisting}[language=golang]
var myProcs []int
var N int

func init() {
	myProcs = []int{1, 3}
	N = 3
}
\end{lstlisting}
\captionof{figure}{Compiled Go}
\end{minipage}

\subsection{Type inference}
PGo will automatically infer types for variables declared in PlusCal. The type inference algorithm supports a limited form of polymorphism to support different use cases for tuples. Specifically, the tuple literal \lstinline[language=pcal]|<<exp1, exp2, exp3>>| may be compiled as a Go slice literal or a Go struct literal depending on whether \lstinline[language=pcal]|exp1|, \lstinline[language=pcal]|exp2|, and \lstinline[language=pcal]|exp3| have the same type.

\subsection{Enabling ModularPlusCal}

In order to use ModularPlusCal, there are two changes to the usual PGo workflow:
\begin{itemize}
    \item In order to model check your ModularPlusCal program, you must first compile it into plain PlusCal in order for it to be interpreted by the TLA+ toolbox. You do this by passing the \lstinline|-m| command-line option to PGo.
    \item TODO: how to enable MPCal to Go compilation?
\end{itemize}

See section \ref{modularpcal} for a description of ModularPlusCal features.

\subsection{Lock inference}
PGo adds locks when compiling a multiprocess PlusCal algorithm. The locking behaviour is described in more detail in~\ref{sec:labels}.
