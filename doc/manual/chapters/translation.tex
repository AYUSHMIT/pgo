PlusCal has many language constructs that are translated to Go in a non-trivial way. These constructs and their translations are described in this section, starting with an overview.

\subsection{Overview}

\subsubsection{PlusCal support}

PGo supports both the C and P-syntaxes of PlusCal. Note that unused labels are removed from the Go output and that fresh variable names and labels are generated in order to avoid name capture.

\begin{center}
\lstset{style=tabularstyle}
\renewcommand{\arraystretch}{1.5}
\begin{longtable}{ || m{0.25\textwidth} | m{0.33\textwidth} | m{0.36\textwidth} || }
	\hline
	\textbf{PlusCal feature} & \textbf{Example code} & \textbf{PGo support} \\
	\hline\hline
	Line comment & \lstinline[language=pcal]|\* line comment| & Supported \\
	\hline
	Block comment & \lstinline[language=pcal]|(* block comment *)| & Supported \\
	\hline
	Labelled statements &
	\begin{lstlisting}[language=pcal]
label:
	stmt1;
	stmt2;
	\* ...
	\end{lstlisting} &
	Compiled with a mutex or a distributed mutex around the statements \\
	\hline
	While loop &
	\begin{lstlisting}[language=pcal]
while (condition) {
	body;
}
	\end{lstlisting} &
	Compiled as
	\begin{lstlisting}[language=golang]
for {
	if !condition {
		break
	}
	body
}
	\end{lstlisting} \\
	\hline
	If statement &
	\begin{lstlisting}[language=pcal]
if (condition) {
	thenPart;
} else {
	elsePart;
}
	\end{lstlisting} &
	Supported; compiled as expected \\
	\hline
	Either statement &
	\begin{lstlisting}[language=pcal]
either { stmt1; stmt2; }
    or { stmt3; stmt4; }
    or { stmt5; stmt6; }
    \* ...
	\end{lstlisting} &
	Compiled as
	\begin{lstlisting}[language=golang]
case0:
	stmt1
	stmt2
	goto endEither
case1:
	stmt3
	stmt4
	goto endEither
case2:
	stmt5
	stmt6
	goto endEither
// ...
endEither:
	\end{lstlisting}

	Each case is tried deterministically from top to bottom (i.e. \lstinline[language=pcal]|case0| is tried before \lstinline[language=pcal]|case1|, etc.). Case N is tried only after case 0 to N-1 have failed because await conditions in those cases are not met.\\
	\hline
	Assignment & \lstinline[language=pcal]|x := exp;| & Supported; compiled as expected \\
	\hline
	Multiple variable assignment & \lstinline[language=pcal]!x := y || y := x + y;! & Supported; compiled as multiple assignment in Go \\
	\hline
	Return statement & \lstinline[language=pcal]|return;| & Supported; compiled as expected \\
	\hline
	Skip statement & \lstinline[language=pcal]|skip;| & Supported; compiled to nothing \\
	\hline
	Call statement & \lstinline[language=pcal]|call procedure1(arg1, arg2);| & Supported; compiled as expected \\
	\hline
	Macro call & \lstinline[language=pcal]|macro1(arg1, arg2);| & Supported; expanded during compilation \\
	\hline
	With statement &
	\begin{lstlisting}[language=pcal]
with (x = exp1, y \in exp2) {
	body;
}
	\end{lstlisting} &
	Supported; compiled as variable assignment with fresh names. In the example code, \lstinline[language=pcal]|y| is assigned the first element of \lstinline[language=pcal]|exp2|.\\
	\hline
	Print statement &
	\lstinline[language=pcal]|print exp;| & Compiled as \lstinline[language=golang]|fmt.Printf("%v", exp)|\\
	\hline
	Assert statement & \lstinline[language=pcal]|assert condition;| &
	Compiled as
	\begin{lstlisting}[language=golang]
if !condition {
	panic("condition");
}
	\end{lstlisting} \\
	\hline
	Await statement & \lstinline[language=pcal]|await condition;| &
	Compiled as
	\begin{lstlisting}[language=golang]
awaitLabel:
	if !condition {
		goto awaitLabel
	}
	\end{lstlisting} \\
	\hline
	Goto statement & \lstinline[language=pcal]|goto label;| & Supported; compiled as expected \\
	\hline
	Single process algorithm &
	\begin{lstlisting}[language=pcal]
--algorithm Algo {
  variables x = exp1, y \in exp2;
  {
  	body;
  }
}
	\end{lstlisting} &
	Supported; compiled as a single-threaded single-process Go program \\
	\hline
	Multiprocess algorithm &
	\begin{lstlisting}[language=pcal]
--algorithm Algo {
  variables x = exp1, y = exp2;
  process (P \in exp3)
  variables local = exp4;
  {
  	body;
  }
}
	\end{lstlisting} &
	Supported; compiled with various strategies configured by the user \\
	\hline
	\caption{PlusCal constructs}
	\label{table:pcal-constructs}
\end{longtable}
\end{center}

\clearpage

\subsubsection{TLA+ support}

Below are the TLA+ constructs. Note that PGo makes liberal use of temporary variables to compile complex TLA+ constructs.

\begin{center}
\lstset{style=tabularstyle}
\renewcommand{\arraystretch}{1.5}
\begin{longtable}{ || m{0.25\textwidth} | m{0.33\textwidth} | m{0.36\textwidth} || }
	\hline
	\textbf{TLA+ feature} & \textbf{Example code} & \textbf{PGo support} \\
	\hline\hline
	Function call &
	\begin{lstlisting}[language=pcal]
x[exp1]
\* or
x[exp1, exp2, exp3]
\* or
x[<<exp1, exp2, exp3>>]
\* or
x[[field1 |-> e1, field2 |-> e2]]
	\end{lstlisting} &
	Supported; compiled code dependent on the type of \lstinline[language=pcal]|x| (the function) \\
	\hline
	Binary operator call & \lstinline[language=pcal]|x /\ y = z + 1| & Supported; compiled as expected \\
	\hline
	Record &
	\begin{lstlisting}[language=pcal]
[field1 \in exp1 |-> exp2,
 field2 |-> exp3]
	\end{lstlisting} &
	Unsupported \\
	\hline
	Function set (as function literal) &
	\begin{lstlisting}[language=pcal]
[Nat -> Nat]
\* or
[Nat -> 1..3]
	\end{lstlisting} &
	Unsupported \\
	\hline
	Function set (as sorted slice of structs) & \lstinline[language=pcal]|[1..3 -> 1..3]| & Unsupported \\
	\hline
	Function substitution &
	\begin{lstlisting}[language=pcal]
[f EXCEPT ![exp1] = exp2]
\* or
[f EXCEPT !.field = exp]
	\end{lstlisting} &
	Unsupported \\
	\hline
	If expression &
	\begin{lstlisting}[language=pcal]
if condition
   then thenExp
   else elseExp
	\end{lstlisting} &
	Compiled as
	\begin{lstlisting}[language=golang]
var result type;
if condition {
	result = thenExp
} else {
	result = elseExp
}
// result is used in place of
// the expression hereafter
	\end{lstlisting} \\
	\hline
	Tuple (as slice) & \lstinline[language=pcal]|<<exp1, exp2, exp3>>| & Compiled as slice when all its contents are of the same type
	\begin{lstlisting}[language=golang]
[]type{exp1, exp2, exp3}
	\end{lstlisting} \\
	\hline
	Tuple (as struct) & \lstinline[language=pcal]|<<exp1, exp2, exp3>>| & Compiled as a struct when at least one element is of a different type from the others' types.

	\begin{lstlisting}[language=golang]
	struct {
		e0 type
		e1 type
		e2 type
	}{exp1, exp2, exp3}
	\end{lstlisting} \\
	\hline
	Case expression &
	\begin{lstlisting}[language=pcal]
CASE x -> y
  [] z -> p
  [] OTHER -> other
	\end{lstlisting} &
	Compiled as
	\begin{lstlisting}[language=golang]
var result type;
if x {
	result = y
	goto matched
}
if z {
	result = p
	goto matched
}
result = other
matched:
// result is used in place of
// the expression hereafter
	\end{lstlisting} \\
	\hline
	Existental &
	\begin{lstlisting}[language=pcal]
\E a, b, c : exp
\* or
\EE a, b, c : exp
	\end{lstlisting} &
	Unsupportable; TLC chokes when given this expression \\
	\hline
	Universal &
	\begin{lstlisting}[language=pcal]
\A a, b, c : exp
\* or
\AA a, b, c : exp
	\end{lstlisting} &
	Unsupportable; TLC chokes when given this expression \\
	\hline
	Let expression &
	\begin{lstlisting}[language=pcal]
LET op(a, b, c) == exp1
    fn[d \in D] == exp2
    e == exp3
IN exp
	\end{lstlisting} &
	Unsupported \\
	\hline
	Assumption &
	\begin{lstlisting}[language=pcal]
ASSUME exp
\* or
ASSUMPTION exp
\* or
AXIOM exp
	\end{lstlisting} &
	Unsupported \\
	\hline
	Theorem & \lstinline[language=pcal]|THEOREM exp| & Unsupported \\
	\hline
	Maybe action & \lstinline[language=pcal]|[exp1]_exp2| &
	Unsupported \\
	\hline
	Required action & \lstinline[language=pcal]|<<exp1>>_exp2| & Unsupported \\
	\hline
	Operator & \lstinline[language=pcal]|Op(arg1, arg2) = exp| & Compiled as a Go function \\
	\hline
	Operator call & \lstinline[language=pcal]|Op(exp1, exp2)| & Supported; compiled as a function call \\
	\hline
	Quantified existential & \lstinline[language=pcal]|\E a \in exp1, b \in exp2 : exp3| &
	Compiled as
	\begin{lstlisting}[language=golang]
exists := false
for _, a := range exp1 {
	for _, b := range exp2 {
		if exp3 {
			exists = true
			goto yes
		}
	}
}
yes:
// exists is used in place of
// the expression hereafter
	\end{lstlisting} \\
	\hline
	Quantified universal & \lstinline[language=pcal]|\A a \in exp1, b \in exp2 : exp3| &
	Compiled as
	\begin{lstlisting}[language=golang]
forAll := true
for _, a := range exp1 {
	for _, b := range exp2 {
		if !exp3 {
			forAll = false
			goto no
		}
	}
}
no:
// forAll is used in place of
// the expression hereafter
	\end{lstlisting} \\
	\hline
	Set constructor & \lstinline[language=pcal]|{exp1, exp2, exp3}| &
	Compiled as sorted slice
	\begin{lstlisting}[language=pcal]
[]type{exp1, exp2, exp3}
	\end{lstlisting} \\
	\hline
	Set comprehension & \lstinline[language=pcal]|{exp : a \in exp1, b \in exp2}| &
	Compiled as
	\begin{lstlisting}[language=golang]
tmpSet := make([]type, 0)
for _, a := range exp1 {
	for _, b := range exp2 {
		tmpSet = append(tmpSet, exp)
	}
}
// more code to ensure elements in
// tmpSet is unique and sorted

// tmpSet is used in place of
// the expression hereafter
	\end{lstlisting} \\
	\hline
	Set refinement & \lstinline[language=pcal]|{a \in exp1 : exp}| &
	Compiled as
	\begin{lstlisting}[language=golang]
tmpSet := make([]type, 0)
for _, v := range exp1 {
	if exp {
		tmpSet = append(tmpSet, v)
	}
}
// tmpSet is used in place of
// the expression hereafter
	\end{lstlisting} \\
	\hline
	\caption{TLA+ constructs}
	\label{table:tla-constructs}
\end{longtable}
\end{center}

\subsection{Variable declarations}
In addition to the simple variable declaration \verb|var = <val>|, PlusCal supports the declaration \verb|var \in <set>|. This asserts that the initial value of \texttt{var} is an element of \texttt{<set>}. This is translated into an assignment of the variable \texttt{var} to the zeroth element of \texttt{<set>}, i.e. \texttt{var = tmpSet[0]}.

\noindent
\begin{minipage}[t]{0.45\textwidth}
\begin{lstlisting}[language=pcal]
variables
	S = {1, 3};
	v \in S;
{
	\* algorithm body...
}
\end{lstlisting}
\captionof{figure}{PlusCal}
\end{minipage}
\hfill
\begin{minipage}[t]{0.45\textwidth}
\begin{lstlisting}[language=golang]
var S []int
var v int

func init() {
	S = []int{1, 3}
	v = S[0]
}

func main() {
	// algorithm body...
}
\end{lstlisting}
\captionof{figure}{Compiled Go}
\end{minipage}

\subsection{Variable assignment}
PlusCal supports multiple variable assignment statements: the statement \texttt{x := a || y := b} evaluates the right-hand sides first, then assigns the values to the left-hand sides. A common use is swapping the variables \texttt{x} and \texttt{y} with the statement \texttt{x := y || y := x}.

Go has a multiple assignment construct, which fits well as a target for this corresponding PlusCal construct.

\noindent
\begin{minipage}[t]{0.45\textwidth}
\begin{lstlisting}[language=pcal]
x := y || y := x + y
\end{lstlisting}
\captionof{figure}{PlusCal}
\end{minipage}
\hfill
\begin{minipage}[t]{0.45\textwidth}
\begin{lstlisting}[language=golang]
x, y = y, x+y
\end{lstlisting}
\captionof{figure}{Compiled Go}
\end{minipage}

\subsection{Macros}
PlusCal macros have the same semantics as C/C++ \texttt{\#define} directives. PGo expands the PlusCal macro wherever it occurs.

\noindent
\begin{minipage}[t]{0.45\textwidth}
\begin{lstlisting}[language=pcal]
variables p = 1, q = 2;
macro add(a, b) {
	a := a + b;
}
{
	add(p, q);
	print p;
}
\end{lstlisting}
\captionof{figure}{PlusCal}
\end{minipage}
\hfill
\begin{minipage}[t]{0.45\textwidth}
\begin{lstlisting}[language=golang]
import "fmt"

var p int = 1
var q int = 2

func main() {
	p = p + q
	fmt.Println("%v", p)
}
\end{lstlisting}
\captionof{figure}{Compiled Go}
\end{minipage}

\subsection{Data types}
PGo supports PlusCal sets, functions, and tuples.
\subsubsection{Sets}
\label{sec:sets}
PlusCal sets are translated into sorted slices in Go.

\noindent
\begin{minipage}[t]{0.45\textwidth}
\begin{lstlisting}[language=pcal]
variables
	A = {1, 2, 3};
	B = {3, 4, 5}
	C = A \union B;
{
	print A = C;
}
\end{lstlisting}
\captionof{figure}{PlusCal}
\end{minipage}
\hfill
\begin{minipage}[t]{0.45\textwidth}
\begin{lstlisting}[language=golang]
A := []int{1, 2, 3}
B := []int{3, 4, 5}
tmpSet := make([]int, len(A), len(A)+len(B))
copy(tmpSet, A)
tmpSet = append(tmpSet, B...)
sort.Ints(tmpSet)
if len(tmpSet) > 1 {
	previousValue := tmpSet[0]
	currentIndex := 1
	for _, v := range tmpSet[1:] {
		if previousValue != v {
			tmpSet[currentIndex] = v
			currentIndex++
		}
		previousValue = v
	}
	tmpSet = tmpSet[:currentIndex]
}
C := tmpSet
eq := len(A) == len(C)
if eq {
	for i := 0; i < len(A); i++ {
		eq = A[i] == C[i]
		if !eq {
			break
		}
	}
}
fmt.Printf("%v\n", eq)
\end{lstlisting}
\captionof{figure}{Compiled Go}
\end{minipage}

PlusCal also supports the typical mathematical set constructor notations.

\noindent
\begin{minipage}[t]{\textwidth}
\begin{lstlisting}[language=pcal]
variables
	S = {1, 5, 6};
	T = {2, 3};
	U = {x \in S : x > 3}; \* equivalent to {5, 6}
	V = {x + y : x \in S, y \in T}; \* equivalent to {3, 4, 7, 8, 9}
\* ...
\end{lstlisting}
\captionof{figure}{PlusCal}
\end{minipage}

\noindent
\begin{minipage}[t]{\textwidth}
\begin{lstlisting}[language=golang]
S := []int{1, 5, 6}
T := []int{2, 3}
tmpSet := make([]int, 0)
for _, x := range S {
	if x > 3 {
		tmpSet = append(tmpSet, x)
	}
}
U := tmpSet
tmpSet0 := make([]int, 0)
for _, x := range S {
	for _, y := range T {
		tmpSet0 = append(tmpSet0, x+y)
	}
}
sort.Ints(tmpSet0)
if len(tmpSet0) > 1 {
	previousValue := tmpSet0[0]
	currentIndex := 1
	for _, v := range tmpSet0[1:] {
		if previousValue != v {
			tmpSet0[currentIndex] = v
			currentIndex++
		}
		previousValue = v
	}
	tmpSet0 = tmpSet0[:currentIndex]
}
V := tmpSet0
// ...
\end{lstlisting}
\captionof{figure}{Compiled Go}
\end{minipage}

While not as concise as the PlusCal, the output Go code is still readable.

\subsubsection{Functions}
TLA+ functions with finite domains are translated into sorted slices of structs in Go.

A TLA+ function can be indexed by multiple indices. This is syntactic sugar for a map indexed by a tuple whose components are the indices.

\noindent
\begin{minipage}[t]{\textwidth}
\begin{lstlisting}[language=pcal]
variables
	S = {1, 2};
	f = [x \in S, y \in S |-> x + y];
	a = f[2, 2]; \* a = 4
\end{lstlisting}
\captionof{figure}{PlusCal}
\end{minipage}

\noindent
\begin{minipage}[t]{\textwidth}
\begin{lstlisting}[language=golang]
S := []int{1, 2}
function := make([]struct {
	key struct {
		e0 int
		e1 int
	}
	value int
}, 0, len(S)*len(S))
for _, x := range S {
	for _, y := range S {
		function = append(function, struct {
			key struct {
				e0 int
				e1 int
			}
			value int
		}{key: struct {
			e0 int
			e1 int
		}{x, y}, value: x + y})
	}
}
f := function
key := struct {
	e0 int
	e1 int
}{2, 2}
index := sort.Search(len(f), func(i int) bool {
	return !(f[i].key.e0 < key.e0 || f[i].key.e0 == key.e0 && f[i].key.e1 < key.e1)
})
a := f[index].value
\end{lstlisting}
\captionof{figure}{Compiled Go}
\end{minipage}

\subsubsection{Tuples}
PlusCal tuples are used in several different contexts, so variables involving tuples may have different inferred types depending on their use. Tuples can store homogeneous data, in which case they correspond to Go slices. Tuple components may be of different types, which correspond to Go structs. PlusCal tuples are 1-indexed, but Go tuples and slices are 0-indexed, so 1 is subtracted from all indices in Go.

\noindent
\begin{minipage}[t]{0.45\textwidth}
\begin{lstlisting}[language=pcal]
variables
	slice = << "a", "b", "c" >>;
	tup = << 1, "a" >>;
{
	print slice[2]; \* "b"
}
\end{lstlisting}
\captionof{figure}{PlusCal}
\end{minipage}
\hfill
\begin{minipage}[t]{0.45\textwidth}
\begin{lstlisting}[language=golang]
slice := []string{"a", "b", "c"}
tup := struct {
	e0 int
	e1 string
}{1, "a"}
fmt.Printf("%v\n", slice[2-1])
\end{lstlisting}
\captionof{figure}{Compiled Go}
\end{minipage}

\subsection{Predicate operations}
PlusCal supports the mathematical quantifiers $\forall$ and $\exists$. PGo compiles these to (nested) for loops, whose bodies check for the relevant condition.

\noindent
\begin{minipage}[t]{\textwidth}
\begin{lstlisting}[language=pcal]
variables
	S = {1, 2, 3};
	T = {4, 5, 6};
	b1 = \E x \in S : x > 2; \* TRUE
	b2 = \A x \in S, y \in T : x + y > 6; \* FALSE
\end{lstlisting}
\captionof{figure}{PlusCal}
\end{minipage}

\noindent
\begin{minipage}[t]{\textwidth}
\begin{lstlisting}[language=golang]
S := []int{1, 2, 3}
T := []int{4, 5, 6}
exists := false
for _, x := range S {
	if x > 2 {
		exists = true
		break
	}
}
b1 := exists
forAll := true
for _, x := range S {
	for _, y := range T {
		if !(x+y > 6) {
			forAll = false
			goto no
		}
	}
}
no:
b2 := forAll
\end{lstlisting}
\captionof{figure}{Compiled Go}
\end{minipage}

\subsection{With}
The PlusCal \texttt{with} statement has the syntax

\noindent
\begin{minipage}[t]{\textwidth}
\begin{lstlisting}[language=pcal]
variables S_1 = {1, 2, 3}, a = "foo";
\* ...
with (x_1 \in S_1, x_2 = a, ...)
{
	\* do stuff with x_i ...
}
\end{lstlisting}
\captionof{figure}{PlusCal}
\end{minipage}

This construct selects the first element from each \texttt{S\_i} and assigns them to the local variables \texttt{x\_i}. If the syntax \texttt{x\_i = a} is used, this simply assigns \texttt{a} to \texttt{x\_i}. In Go, this translates to

\noindent
\begin{minipage}[t]{\textwidth}
\begin{lstlisting}[language=golang]
S_1 := []int{1, 2, 3}
a := "foo"
x_1 := S_1[0]
x_2 := a
\end{lstlisting}
\captionof{figure}{Compiled Go}
\end{minipage}

The local variables declared by the \texttt{with} and its body are potentially renamed to ensure no accidental capture.

\subsection{Processes}
The PlusCal algorithm body can either contain statements in a uniprocess algorithm, or process declarations in a multiprocess algorithm.

Uniprocess algorithms are translated into single-threaded Go programs.

In a multiprocess algorithm, processes can be declared with the syntax \verb|process (Name \in S)| or \verb|process (Name = Id)|. The first construct spawns a set of processes, one for each ID in the set \texttt{S}, and the second spawns a single process with ID \texttt{Id}. A process can refer to its identifier with the keyword \texttt{self}.

The following is a simple example of a multiprocess PlusCal algorithm, which will be used as the translation source throughout this subsection:

\noindent
\begin{minipage}[t]{\textwidth}
	\begin{lstlisting}[language=pcal]
variables
	idSet = {1, 2, 3};
	id = "id";

process (PName \in idSet)
variable local;
{
	local := self;
}

process (Other = id) {
	print self;
}
	\end{lstlisting}
	\captionof{figure}{PlusCal}
\end{minipage}

\subsubsection{Multi-threaded compilation strategy}
With the multi-threaded compilation strategy, PGo converts each process body to a function and spawns a goroutine per process. The function takes a single parameter \texttt{self}, the process ID. There are two semantic considerations: the main goroutine should not exit before all goroutines finish executing, and the time at which all child goroutines begin executing should be synchronized. To preserve these semantics, PGo uses a global waitgroup which waits on all goroutines, and each process body pulls from a dummy channel before beginning execution. When all goroutines have been initialized, the dummy channel is closed so that the channel pull no longer blocks, allowing for a synchronized start.

Below is the output Go program when compiled using the multi-threaded compilation strategy. Note that all processes use the same waitgroup (\texttt{PGoWait}) and dummy channel (\texttt{PGoStart}).

\noindent
\begin{minipage}[t]{\textwidth}
\begin{lstlisting}[language=golang]
package main

import (
	"fmt"
	"sync"
)

var idSet []int

var id string

var pGoStart chan bool

var pGoWait sync.WaitGroup

func init() {
	idSet = []int{1, 2, 3}
	id = "id"
	pGoStart = make(chan bool)
}

func PName(self int) {
	defer pGoWait.Done()
	<-pGoStart
	local := 0
	local = self
}

func Other(self string) {
	defer pGoWait.Done()
	<-pGoStart
	fmt.Printf("%v\n", self)
}

func main() {
	for _, v := range idSet {
		pGoWait.Add(1)
		go PName(v)
	}
	pGoWait.Add(1)
	go Other(id)
	close(pGoStart)
	pGoWait.Wait()
}
\end{lstlisting}
\captionof{figure}{Compiled multi-threaded Go program}
\end{minipage}

\subsubsection{Distributed process compilation strategy}

TODO

\subsection{Labels}
\label{sec:labels}
In PlusCal, labels are used as targets for \texttt{goto} statements and also to specify atomic operations. If a statement is labelled, all statements up to, and excluding, the next label are considered to be a single atomic operation. In Go, unused labels cause compilation errors so PGo only outputs labels when they are targets of some \texttt{goto} statement.

To deal with atomicity, PGo divides the global variables into groups and guards each group with a \texttt{sync.RWMutex}. PGo groups variables by performing a set union, merging two variable sets when two variables in them can be accessed in the same label. The following is a simple example:

\noindent
\begin{minipage}[t]{0.45\textwidth}
\begin{lstlisting}[language=pcal]
variables a = 0, b = 1, c = 2, d = 3;
process (P \in {1, 2, 3}) {
lab1:   a := 1;
        b := 2;
lab2:   b := 3;
        c := 4;
lab3:   d := 5;
}
\end{lstlisting}
\captionof{figure}{PlusCal}
\end{minipage}
\hfill
\begin{minipage}[t]{0.45\textwidth}
\begin{lstlisting}[language=golang]
package main

import (
"sync"
)

var a int

var b int

var c int

var d int

var pGoStart chan bool

var pGoWait sync.WaitGroup

var pGoLock []sync.RWMutex

func init() {
	a = 0
	b = 1
	c = 2
	d = 3
	pGoStart = make(chan bool)
	pGoLock = make([]sync.RWMutex, 2)
}

func P(self int) {
	defer pGoWait.Done()
	<-pGoStart
	pGoLock[0].Lock()
	a = 1
	b = 2
	pGoLock[0].Unlock()
	pGoLock[0].Lock()
	b = 3
	c = 4
	pGoLock[0].Unlock()
	pGoLock[1].Lock()
	d = 5
	pGoLock[1].Unlock()
}

func main() {
	for _, v := range []int{1, 2, 3} {
		pGoWait.Add(1)
		go P(v)
	}
	close(pGoStart)
	pGoWait.Wait()
}
\end{lstlisting}
\captionof{figure}{Compiled Go}
\end{minipage}
The variable \texttt{b} may be accessed atomically with \texttt{a} (in the label \texttt{lab1}) and also with \texttt{c} (in the label \texttt{lab2}) so all three of \texttt{a}, \texttt{b}, and \texttt{c} must be grouped together to prevent data races. PGo locks the correct group before each atomic operation and unlocks it afterwards. Even single operations must use the lock, since there are no atomicity guarantees for most Go statements. If the atomic operations are specified to be smaller in PlusCal by adding more labels, PGo will compile smaller variable groups, allowing for more parallelism.

\subsection{Limitations}
Not all PlusCal specifications can be compiled by PGo. This is an overview of some important PlusCal features that are currently unsupported.

\begin{itemize}
\item Referencing \lstinline[language=pcal]|self| in a procedure call

\item TLA+ features:

	\begin{itemize}
	\item Alignment of boolean operators in bulleted lists determining precedence
	
	\item Record sets
	
	\item Bags (multisets)
	
	\item The \lstinline[language=pcal]|LET .. IN| construct
	
	\item Temporal logic operators
	
	\item Recursive definitions
	\end{itemize}
\end{itemize}

PGo does not yet have a coherent story for the following desirable features for programmers. However, work on Modular PlusCal which aims to support them is underway.

\begin{itemize}
\item Interfacing with other programs

\item Reading input from the outside world (e.g. from the command line, from the disk)
\end{itemize}
