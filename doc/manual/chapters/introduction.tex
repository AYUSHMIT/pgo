TLA+ is a formal specification language, built on the mathematical concepts of set theory and the temporal logic of actions. Using the TLC model checker, TLA+ specifications can be checked exhaustively for specific properties, and the TLA proof system (TLAPS) allows for machine-checked proofs. PlusCal is an algorithm language which can be translated to TLA+ and uses TLA+ as its expression language. It is easy to specify and verify distributed algorithms in PlusCal, thanks to its simple constructs for nondeterminism, concurrency primitives, and rich mathematical constructs. However, PlusCal does not correspond well to a real implementation. The TLA proof system itself does not provide any way of extracting executable code from a PlusCal algorithm, nor does the PlusCal language provide any facilities for describing the kind of interface that extracted code should provide.

PGo aims to correspond a verified PlusCal specification with an executable Go implementation. PGo either compiles a PlusCal algorithm into a corresponding Go implementation with no interface, or accepts a superset of PlusCal called ModularPlusCal that can be compiled into a free-standing Go module that may be used as a library.

Since implementation details like network communication and environmental non-determinism cannot be written in the specification, PGo generates code that is parameterised on implementations of those things. It is then possible to invoke the compiled algorithm using either stock implementations provided by PGo's runtime library or, should the need arise, any implementations that match interface provided.