TLA+ is a formal specification language, built on the mathematical concepts of set theory and the temporal logic of actions. Using the TLC model checker, TLA+ specifications can be checked exhaustively for specific properties, and the TLA proof system (TLAPS) allows for machine-checked proofs. PlusCal is an algorithm language which can be translated to TLA+ and uses TLA+ as its expression language. It is easy to specify and verify distributed algorithms in PlusCal, thanks to its simple constructs for nondeterminism, concurrency primitives, and rich mathematical constructs. However, PlusCal is not a programming language so it cannot be run, which limits its utility.

PGo aims to correspond a verified PlusCal specification with an executable Go implementation. PGo compiles an annotated PlusCal specification into a Go program which preserves the semantics of the specification. The main goal of PGo is to create a compiled Go program which is easy to read and edit. The intended use case for PGo is for the developer to create and verify a PlusCal spec of the system, compile it with PGo, then edit the compiled program to fully implement details not included in the specification. Programs compiled by PGo are type safe (they do not contain type assertions which panic).